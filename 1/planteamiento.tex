\chapter{PLANTEAMIENTO DEL PROBLEMA}
\section{Descripción de la Realidad Problemática}

La protección de la propiedad intelectual es un pilar fundamental para la innovación y el desarrollo económico. Las marcas registradas, en particular, juegan un rol crucial al identificar el origen comercial de productos y servicios, protegiendo así al consumidor y al propietario de la marca. Sin embargo, el proceso de registro y verificación de estas marcas enfrenta desafíos significativos, especialmente cuando involucra elementos figurativos o gráficos en lugar de solo texto.

Desafíos en la detección y registro de marcas figurativas:

Volumen y variabilidad: Cada año se presentan millones de solicitudes de marcas en todo el mundo. Según la Organización Mundial de la Propiedad Intelectual (OMPI), en 2019 se registraron aproximadamente 11.5 millones de marcas activas solo en sus 55 jurisdicciones miembros. Esta enorme cantidad hace manualmente impracticable la revisión exhaustiva de cada nueva solicitud en búsqueda de posibles conflictos con marcas existentes.
Subjetividad en la interpretación: La evaluación de similitudes entre elementos gráficos es altamente subjetiva y depende en gran medida del criterio del examinador, lo que puede llevar a inconsistencias en la protección de las marcas.
Evolución tecnológica: Con el avance de las tecnologías digitales, el aumento de marcas que incluyen elementos digitales y multimedia es notable, complicando aún más el proceso de verificación y búsqueda de antecedentes.
Impacto económico y legal:

Pérdida de ingresos y litigios: La falta de una detección eficaz puede resultar en conflictos legales prolongados y costosos por derechos de marca, además de pérdidas económicas significativas tanto para las empresas como para los consumidores.
 

\section{Formulación del Problema}

Para lefecto \parencite{ot_marti2018manual}. 


Una vez elaborado el diagrama (véase Anexo 1), 

\subsection{Problema General}
\newcommand{\ProblemaGeneral}{
	¿Cómo mejorar la eficiencia y precisión en el proceso de detección y verificación de marcas registradas que contienen elementos gráficos mediante la implementación de modelos de Deep Learning en la búsqueda figurativa de imágenes?
}
\ProblemaGeneral
\subsection{Problemas Espec\'{i}ficos}
\newcommand{\Pbone}{
¿Cómo pueden los modelos de Deep Learning ayudar a reducir el tiempo y el costo asociados con la búsqueda de antecedentes de marcas registradas figurativas?
}
\newcommand{\Pbtwo}{
¿De qué manera los modelos de Deep Learning pueden minimizar la subjetividad en la evaluación de similitudes visuales entre marcas registradas?
}
\newcommand{\Pbthree}{
¿Cuáles son los principales desafíos técnicos y legales en la implementación de sistemas automatizados para la detección de marcas registradas?
}

\begin{itemize}
	\item \Pbone
	\item \Pbtwo
	\item \Pbthree
\end{itemize}

\section{Objetivos de la Investigación}
Para la formulación de los objetivos de la presente investigación se elaboró un «árbol de objetivos» (véase Anexo 2) 
\subsection{Objetivo General}
\newcommand{\ObjetivoGeneral}{
Desarrollar e implementar un modelo de Deep Learning eficiente y efectivo para la detección automatizada de marcas registradas mediante la búsqueda figurativa de imágenes, que mejore la precisión y reduzca el tiempo y los costos del proceso de registro y verificación.
}
\ObjetivoGeneral
\subsection{Objetivos Espec\'{i}ficos}
\newcommand{\Objone}{
Diseñar un modelo de Deep Learning que identifique con alta precisión similitudes y diferencias entre marcas registradas que incluyan elementos gráficos.
}
\newcommand{\Objtwo}{
Evaluar la capacidad del modelo propuesto para operar dentro de los marcos legales y normativos existentes relacionados con las marcas registradas.
}
\newcommand{\Objthree}{
Implementar una prueba de concepto del modelo desarrollado para demostrar su aplicabilidad y beneficios en un entorno real de registro de marcas.
}

\begin{itemize}
	\item {\Objone}
	\item {\Objtwo}
	\item {\Objthree}
\end{itemize}

\section{Justificación de la Investigación}

\subsection{Teórica}
Esta investigación se realiza 

\subsection{Práctica}
Al culminar la investigación 

\subsection{Metodológica}. 

\section{Delimitación del Estudio}

\subsection{Espacial}
Para la presente investigación 

\subsection{Temporal}
Los datos que serán necesari. 

\subsection{Conceptual}
Esta investigación se 

\section{Hipótesis}

\subsection{Hipótesis General}
\newcommand{\HipotesisGeneral}{
El uso de técnicas de.
}
\HipotesisGeneral
\subsection{Hipótesis Específicas}
\newcommand{\Hone}{
	x
}
\newcommand{\Htwo}{
	y
}
\newcommand{\Hthree}{
	z	
}
\newcommand{\Hfour}{
	cv
}
\newcommand{\Hfive}{
	xws
}
\begin{itemize}
	\item \Hone
	\item \Htwo
	\item \Hthree
	\item \Hfour
	\item \Hfive
\end{itemize}

\subsection{Matriz de Consistencia}
A continuación se presenta la matriz de consistencia elaborada para la presente investigación (véase Anexo \ref{1:table}).

